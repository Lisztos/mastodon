\chapter*{Zusammenfassung}
\label{cha:zusammenfassung}

Die zentralisierte Architektur, die das derzeitige soziale Web kennzeichnet, führte zur Entwicklung dezentraler sozialer Online-Netzwerke (DOSN). Diese DOSN stützen sich auf ihre Kommunikationsprotokolle, um miteinander interagieren zu können, was die Schaffung eines grenzenlosen Ökosystems ermöglicht, in dem sich die Benutzer nahtlos zwischen Anwendungen bewegen können, ohne ihr Netzwerk von Freunden und Profilen an jedem Zielort neu aufbauen zu müssen.

ActivityPub ist ein standardisiertes dezentralisiertes Protokoll für soziale Netzwerke, das von einer großen Anzahl von DOSNs implementiert wird. Dieser Standard hat es jedoch versäumt, einen Weg zu definieren, wie die Kommunikation sicher, vertraulich, privat und nicht widerlegbar sein kann. Darüber hinaus bieten seine Implementierungen keine dezentralisierte Möglichkeit zur Verwaltung von Identitäten.

Das Web 3.0 bringt Dezentralisierung in die derzeitige, größtenteils siloartige Web-Architektur, und zwar durch neue Technologien, die neue Standards ermöglichen, die das Potenzial haben, die Art und Weise, wie das Web genutzt wird, zu verändern. Die kürzlich standardisierten \emph{Decentralized Identifiers} (DIDs) des W3C und das DID-basierte Kommunikationsprotokoll \emph{DIDComm Messaging v2} sind zwei Spezifikationen, die eine Dezentralisierung der Identitätsverwaltung und der Kommunikationssicherheit versprechen.  

Diese Arbeit schlägt vor, die Einschränkungen von ActivityPub-basierten sozialen Netzwerken zu beheben, indem DIDs integriert werden, um ein dezentralisiertes Identitätsmanagement zu ermöglichen, und indem DIDComm einen dezentralen Weg für sichere Kommunikation bietet.