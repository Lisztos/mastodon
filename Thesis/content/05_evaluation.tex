\chapter{Evaluation}
\label{cha:evaluation}

% The type and setup of your evaluation strongly depends of the type of project. Discuss this early on with your supervisor.
% DO NOT UNDERESTIMATE this chapter! It is a crucial part of your work (contribution). It will take up a lot (!) of time and effort! Start early!
% Generally, you should compare your approach to other approaches, so your contribution is quantifiable.
% Explain in detail, what you evaluate and how you evaluate it. Give all details necessary for a reproduction of the results. The idea is that if someone else would evaluate your work with the exact setup you describe here, he/she should end up with the same results!
% Describe and discuss all results. Don't omit the raw data.
% Depending on the type of thesis and evaluation, this can be anything between 10 and 30 pages.

The design for a DIDComm-enabled ActivityPub protocol for federated social networks strives to take advantage of the features that both DIDs and DIDComm provide. The following chapter evaluates the proposed design to assess if these features were achieved. In addition, the evaluation compares the state-of-the-art ActivityPub implementation, the extended versions mentioned in \ref{sec:extending_activitypub} and the proposed design.

\section{Decentralization}
\subsection{Trust Encryption}
ActivityPub relies on HTTPS to provide confidentiality and data integrity. This comprises a dependence on the authority issuing the certificates that make the TLS/SSL encryption possible. By implementing DIDComm, the encryption trust gets transferred to the decentralized identifiers. Thus removing any dependence from any third parties. 

\subsection{Resolving Process}
The current implementation of Mastodon relies on the Webfinger protocol, which itself relies on the DNS, to resolve \emph{username@domain}-formatted usernames to an actor object. The DNS builds the foundation of daily Internet activity. As a centralized service, being so involved in the critical part of the Internet makes it a valuable target for attacks, such as \emph{spoofing}, \emph{ID hacking} or \emph{cache poisoning} \cite{carli2003security}. A single institution controls it, the ICANN\footnote{https://www.icann.org/}, that until 2016 was under the control of one single country \cite{lee_2016}. 
By integrating ledger-based DIDs, this dependency on Webfinger and the DNS was only partially removed. Resolving a DID does not require the DNS to be successful. However, the service endpoint in the DID document still includes the domain of the Mastodon instance where it resides. To retrieve the actor object of the user being looked up, an HTTP GET request to the service endpoint is necessary. This means that a DNS resolution will still take place. On the contrary, the approach of the coauthors of ActivityPub and the DIDs mentioned in \ref{sec:extending_activitypub} completely removes the DNS dependency by using The Router Onion or \emph{Tor} to host the ActivityPub server \cite{webber_sporny_2017}. This indicates that if Mastodon were running in a DNS-free space, the design would achieve a fully-decentralized resolving process. 
 
\subsection{Identity Management}
Mastodon creates accounts for users using basic password authentication. A federated way to create an account using a third party like \emph{Google} or \emph{Facebook} has not been implemented. Further attempts to improve identity verification have been made, such as using \emph{Keybase}\footnote{https://keybase.io} to \emph{verify} the account. Keybase offers a platform that allows prooving ownership of a social network account, like Twitter or Reddit, with the help of public key cryptography. However, this functionality was removed soon after Zoom\footnote{https://zoom.us} bought it in 2020. According to Mastodon's founder, a user could still use any name in Keybase, and it was still unclear if this person was whom she said she was \cite{rochko_2020}.

As mentioned in \ref{subsec:mastodon}, Mastodon uses basic password authentication to create accounts. With this, the account is restricted to this specific server. The user has no other way to register to a different server, and the identity lives as long as the instance keeps running. On the contrary, in the proposed design, the use of DIDs brings an SSI approach where the user creates his identity independently from Mastodon. In the design, the DIDs are anchored to a test network of Ethereum. Consequently, their existence and validity will persist even if the Mastodon instance ceases to exist, eliminating the \emph{single-point-of-failure} of the identity itself.  

\section{Security}
Mastodon implemented HTTP Signatures to add non-repudiation and preserve integrity against tampering to the messages sent within the federation. This approach fulfills the same goals as the JWS token used in the proposed design. Both methods follow the same idea and use public key cryptography to perform the signature. Furthermore, the same key generation algorithm is being used by both. Nonetheless, the downside of HTTP signatures is that they are limited to the HTTP protocol, whereas the JWS has no constraints. 
Moreover, the JSON-LD signatures provide an HTTP-independent manner to provide non-repudiation. However, the superiority of one over the other is still an open discussion \footnote{https://w3c.github.io/vc-imp-guide/\#benefits-of-json-ld-and-ld-proofs}. JSON-LD signatures offer more flexibility and thus scalability for global decentralized networks due to their compatibility with JSON-LD. In contrast, JWTs provide a straightforward way to express data with low overhead \cite{chadwick_longley_sporny_terbu_zagidulin_zundel_2022}. As Mastodon only implements them in particular cases, it is not possible to reach a verdict. 
For this thesis, the conclusion is that both the Mastodon and the proposed implementation offer the same level of non-repudiation when using HTTP as the transport protocol. 

\section{Privacy}
The user has no control over his data by using centralized identity management. DIDs, on the other hand,  implement natively the 7 Foundational Principles of \emph{Privacy by Design} \cite{cavoukian_2006}. This gives consumers control over their personal information, including choosing what to disclose and what not. \cite{sporny_longley_sabadello_reed_steele_2021}. The proposed design requires the profile URL of the user to be included in the DID document to allow the decentralized discovery of his actor object. Although the parameter inside the profile URL does not include user's personal information, like a human-friendly username, anyone can still access this endpoint and get all the information from the actor object. This endpoint raises a significant privacy concern and opens the question of how to be discoverable while keeping privacy. A possibility could be adding the inbox URL instead of the profile URL in the DID document. This possibility would provide a direct endpoint to interact via ActivityPub with the user. This approach might prove to be better than the proposed design, not only by improving privacy but also by shortening the number of requests needed to resolve an account and find the inbox URL for the use case defined in \autoref{section:use_case}. Nonetheless, there are many other use cases where having just the inbox URL might be limiting. For example, when \emph{Alice} wants to follow \emph{Bob}. An extra service endpoint with the \emph{follow} URL would be necessary to achieve this with the same approach, and so on with other endpoints until we get to the same approach of Webber and Sporny, mentioned in \autoref{sec:extending_activitypub}. 


\section{Confidentiality}
As explained before, the only confidentiality provided by Mastodon's implementation is the use of HTTPS. By enabling DIDComm, it is possible to encrypt the payload before sending it and decrypt it after it has arrived at its target server, extending the scope of confidentiality beyond the transport layer. Theoretically, only the DIDs at the respective endpoints could see the plain text message. 
Nonetheless, the confidentiality level reached by DIDComm and the JWE standard in the proposed design has not yet achieved complete end-to-end encryption. The reason is that the Mastodon instance must have access to the private key of the DID subject in order to decrypt the payload and display the message to the user, as mentioned in the DIDComm requirements in \autoref{section:enabling_didcomm}. This imposes a significant risk because the private key is no longer under the user's control. The administrator of the Mastodon instance would have access to the plain-text private key, and some security countermeasures like key rotation would not be able to counter this. 



