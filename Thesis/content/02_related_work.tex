\chapter{Related Work}
\label{cha:relatedwork}

The following chapter covers the most significant concepts required to comprehend this thesis's approach. It includes a closer look at decentralized communication protocols, identifier standards, and social networks that implement them. The revision and structuring of these concepts allow us to understand, build upon, and apply them to address our identified research questions.

\section{Social Web Protocols}
Between 2014 and 2018, the Social Web Working Group (SocialWG) from the W3C embarked on the journey to bring social-networking standards to the Web. This journey included defining technical protocols, vocabularies, and APIs focusing on social interactions. In addition, systems implementing these features should be able to communicate with each other in a decentralized manner. These four years resulted in several W3C Recommendations, including a collection of standards that enable various aspects of decentralized social interaction on the Web called \emph{Social Web Protocols}.\cite{celik_prodromou_le_hors_2014}. Standards found in this collection are \emph{WebSub}\footnote{https://www.w3.org/TR/websub/}, \emph{WebMention}\footnote{https://www.w3.org/TR/webmention/}, \emph{Linked Data Notifications}\footnote{https://www.w3.org/TR/ldn/}, and the two most relevant for this thesis, \emph{ActivityStreams 2.0}\footnote{https://www.w3.org/TR/activitystreams-core/} and \emph{ActivityPub}\footnote{https://www.w3.org/TR/activitypub/}.

% ------------ActivityStreams------------------------------------
\subsection{ActivityStreams 2.0}\label{subsec:activitystreams}
First, it is essential to be able to describe an Activity. ActivityStreams 2.0 is a standard that provides a model for representing \emph{Activities} using a JSON-based syntax. Additionally, it provides a vocabulary that includes all the standard terms needed to represent social activities \cite{snell_prodromou_2017}. This standard describes an activity following a story of \emph{an actor performing an action on an object}. For this, it specifies different types of actors, activities, and objects, as shown in \autoref{table:activitystreams_vocabulary}. Each of these objects can be represented as a JSON object, creating a solid foundation upon which other protocols can build. 

\begin{table}[H]
  \centering
  \begin{tabular}{|p{4cm}|p{4cm}|p{4cm}| }
    \hline
    \multicolumn{3}{|c|}{ActivityStreams Vocabulary} \\
    \hline
    Activity types & Actor types & Object types \\
    \hline
    \hline
      Accept, Add & Application & Note \\
      Announce, Arrive & Group & Document \\ 
      Block, Create & Organization & Image \\
      Delete, Dislike & Person & Article \\
      Flag, Follow & Service & Profile \\
      Ignore, Invite & & Audio \\
      Join, Leave & & Event \\
      Like, Listen & & Tombstone \\
      etc... & & etc... \\
      \hline
  \end{tabular}
  \caption{ActivityStreams 2.0 vocabulary examples}
  \label{table:activitystreams_vocabulary}
\end{table}

ActivityStreams 2.0 has improved its 1.0 version in more than one perspective. One of these is the compatibility with JSON-LD\footnote{https://www.w3.org/TR/json-ld/}, which is a JSON serialization for \emph{Linked Data}\footnote{https://www.w3.org/DesignIssues/LinkedData.html}. The concept of Linked Data is based on interlinking data in such a way that it becomes more usable through associative and contextual queries \cite{berners-lee_2006}.\. With JSON-LD, ActivityStreams 2.0 can define its own context and the terms that will be used inside this context. Figure \ref{fig:activitystream_example} shows an example of a JSON-LD serialized ActivityStreams 2.0 activity. 

\lstset{style=JSONStyle}
\begin{lstlisting}[language=PHP, caption=Example of activity \cite{snell_prodromou_2017}, label=fig:activitystream_example, float=h]
  {
    "@context": "https://www.w3.org/ns/activitystreams",
    "summary": "Alice created an image",
    "type": "Create",
    "actor": "http://www.test.example/Alice",
    "object": "http://example.org/foo.jpg"
  }
\end{lstlisting}

% ------------ActivityPub------------------------------------
\subsection{ActivityPub}\label{subsec:activitypub}

ActivityPub is another W3C Recommendation that originated from the SocialWG. It is a decentralized social networking protocol that is based on the syntax and vocabulary of ActivityStreams 2.0. It provides a client-to-server API, which covers the requirements of a Social API\cite{guy_2017}, i.e., publishing, subscribing, reading content, and notifying when content gets created. In addition, it provides a server-to-server API that enables federated communication. Furthermore, it provides users with a JSON-based \emph{profile}, which is an ActivityStreams 2.0 actor object. This actor object includes standard properties such as \emph{name}, \emph{type}, and \emph{summary}. ActivityPub extended this actor object with several properties, as shown in \ref{fig:actor_object}. Optional properties include collections such as \emph{following, followers, liked} and compulsory properties include an \emph{inbox} and an \emph{outbox}. These last two are URLs that represent how the actor gets and sends messages from other users.


%  Actor Object ActivityPub
\lstset{style=JSONStyle}
\begin{lstlisting}[language=PHP, caption=Actor object example in ActivityPub \cite{lemmer-webber_tallon_guy_prodromou_2018}, label=fig:actor_object, float=h!]
  {
    "@context": "https://www.w3.org/ns/activitystreams",
    "type": "Person",
    "id": "https://social.example/alice/",
    "name": "Alice P.",
    "preferredUsername": "alice",
    "summary": "TU Berlin student",
    "inbox": "https://social.example/alice/inbox/",
    "outbox": "https://social.example/alice/outbox/",
    "followers": "https://social.example/alice/followers/",
    "following": "https://social.example/alice/following/",
    "liked": "https://social.example/alice/liked/"
  }
\end{lstlisting}

There are two workflows of communication for a user in ActivityPub: \label{subsec:ap_workflows}

\begin{itemize}
  \item \textbf{Client-to-server communication:} A user wants to share a post so it makes an HTTP POST request to its outbox with the respective activity object. After this, other users interested in seeing this user's posts can make an HTTP GET request to the user's outbox and retrieve all his public posts.
  \item \textbf{Server-to-server communication (Federation)}: User \emph{A} wants to send a post to user \emph{B}, whose account is on a different server. First, user \emph{A} posts his message to his outbox. Consequently, his server looks for \emph{B}'s inbox and performs an HTTP POST request. Finally, \emph{B} makes an HTTP GET request to his inbox to retrieve all the posts addressed to him. A key thing to remember is that for this type of communication, \emph{A}'s server has to retrieve somehow the \emph{inbox} of user \emph{B} based only on his username. This resolving process is not part of the ActivityPub specification, therefore,  implementers of this standard have to figure out how to do it independently. 
\end{itemize}
 

\begin{figure}[H]
  \centering
  \includegraphics[width=\textwidth]{related_work/ActivityPub-tutorial-image.png}
  \caption{ActivityPub overview \cite{lemmer-webber_tallon_guy_prodromou_2018}}
  \label{fig:ap_flow}
\end{figure}
\pagebreak

Regarding security, ActivityPub's specification does not define any official security mechanisms to ensure confidentiality, non-repudiation, message integrity, authentication, or authorization \cite{lemmer-webber_tallon_guy_prodromou_2018}. However, it mentions a list published by the SocialWG of best security practices\footnote{https://www.w3.org/wiki/SocialCG/ActivityPub/Authentication\_Authorization} that may be used in ActivityPub. This list suggests using standards such as OAuth 2.0\footnote{https://oauth.net/2/} for client-to-server authentication, as well as HTTP Signatures\footnote{https://tools.ietf.org/html/draft-cavage-http-signatures-08} and Linked Data Signatures\footnote{https://w3c-dvcg.github.io/ld-signatures/} for server-to-server authentication. Furthermore, it recommends using HTTPS for its HTTP-based communication to provide at least transport-layer encryption.


% ------------ActivityPub-Based social networks------------------------------------
\section{ActivityPub-based Social Networks}

\subsection{The Fediverse}

It is impossible not to refer to the \emph{Fediverse} when we talk about ActivityPub. The \emph{Fediverse} is an interoperable collection of different federated social networks running on free open software on thousands of servers across the world that implement the same open-standard protocols to be able to interact with each other. In today's most popular social networks like Facebook, Twitter, or Youtube, a centralized architecture keeps millions of users on one platform. Control, decision-making, user data, and censorship depend on a single profit-driven organization. On the contrary, the \emph{Fediverse} is developed by a not-profit-driven community of people around the globe independent of any corporation or official institution \cite{holloway_2018} \cite{https://doi.org/10.48550/arxiv.1909.05801}. The simplest way to explain how the federation works is the following example: Bob has a Twitter account, which he uses to follow all his friends that also have a Twitter account. Alice is a friend of Bob, but she only has an account on Youtube. In the real world, these two services are completely isolated and cannot communicate with each other. However, if both had implemented the same social network protocol, such as ActivityPub, Bob would be able to find Alice by a normal search on Twitter and follow her. Allowing any new post of Alice on Youtube, to appear in Bob's Twitter timeline. \\
Before ActivityPub, the \emph{Fediverse} implemented other protocols like \emph{Ostatus}\footnote{https://www.w3.org/community/ostatus/wiki/images/9/93/OStatus\_1.0\_Draft\_2.pdf}, \emph{Matrix}\footnote{https://matrix.org}, and \emph{Diaspora}\footnote{https://diaspora.github.io/diaspora\_federation}. However, after ActivityPub was published as a Recommendation by the SocialWG in January 2018, a big number of these federated social networks upgraded to ActivityPub. Becoming rapidly the predominant protocol. Furthermore, the range of services that can be found inside the \emph{Fediverse} includes blogging, microblogging, video streaming, photo, music sharing as well as file hosting. For example: 

\begin{itemize}
  \item \textbf{PeerTube\footnote{https://joinpeertube.org}:} A decentralized alternative to video platforms, similar to Youtube.
  \item \textbf{Mastodon\footnote{https://joinmastodon.org}:} A microblogging social network, similar to Twitter. 
  \item \textbf{Pixelfed\footnote{https://pixelfed.org}:} An image-sharing platform, similar to Instagram. 
\end{itemize} 

Although it was not the first social network to implement ActivityPub, Mastodon is the one that pioneered its use on a large scale \cite{lemmer-webber_2017}. Moreover, it is still the social network in the \emph{Fediverse} with the biggest user base and popularity. For this reason, Mastodon will be used in this thesis as the representative of the ActivityPub-based social networks and will be explained further in the next section. 

% ------------Mastodon------------------------------------
\subsection{Mastodon}\label{subsec:mastodon}

 Mastodon is a decentralized Twitter-like microblogging social network created with the idea to bring social networking back into the hands of its users. The german creator of Mastodon, Eugen Rochko, shared the same opinion as what Fitzpatrick and Recordon said in 2007 \cite{fitzpatrick_recordon_2007}: \emph{People are getting sick of registering and re-declaring their friends on every site}. For this reason, Eugen envisioned a social network that could end this, and \emph{last forever} \cite{tilley_2018}. Similar to the \emph{Fediverse}, Mastodon differs from other commercial social networks in two aspects. First, it is oriented towards small communities and community-based services. Each \emph{instance}\footnote{A server running Mastodon} is free to choose its topics, this way users are encouraged to choose the instance better suited to their taste. Second, the Mastodon platform eliminates the presence of sponsored users or posts in feeds. This implies that the only way to connect or consume content is through a self-search to find an already known account or to explore the feeds of the users in other instances with similar interests \cite{8845221}. From a user experience perspective, Mastodon includes all the essential features of a microblogging platform, such as:

\begin{itemize}
  \item Follow other users, even if they are not in the same instance. 
  \item Post small status updates, or \emph{toots}, up to 500 characters long. 
  \item Access to a timeline of the local instance and federated statuses. 
  \item Control over the visibility of their posts, with the option to set them as private, instance-level only, or federated. 
\end{itemize}

Mastodon's implementation of ActivityPub follows the guidelines defined by the spec. However, the protocol does not specify how to implement some key processes required for a fully-working social network. Therefore, Mastodon extended and complemented the protocol with the following processes and features.


\subsubsection{Security}
Mastodon has implemented the authentication and authorization mechanisms from the best practices list of the SocialWG, to address the security concerns in ActiviyPub. However,  relevant for this thesis are the ones used in federated communication, namely HTTP and JSON-LD Signatures. HTTP signatures extend the HTTP protocol by adding the possibility to cryptographically sign the HTTP requests. This signature gets added to the request within the \emph{Signature} header, and it provides not only end-to-end message integrity but also proof of the authenticity of the sender without the need for multiple round-trips \cite{cavage_sporny_2019}. Creating an HTTP signature means signing the parameters of the request itself, i.e the \emph{request-target}, the \emph{host}, and the \emph{date}. These parameters and their values are concatenated in a single string, hashed, and signed with the sender's public key. Finally, the \emph{Signature} header indicates the key that was used to sign the document, the parameters that are inside the signature, and the algorithm used to hash it. An example in Mastodon is shown by figure \ref{fig:http_signature}. 
Following the same idea, Linked Data Signatures offer a way to create and attach signatures to JSON-LD documents, thus providing non-repudiation to e.g. an ActivityStreams Activity object even if the object has been shared, forwarded, or referenced at a future time \cite{celik_prodromou_le_hors_2014}. However, this feature, although implemented, is not actively used in Mastodon.

\lstset{style=JSONStyle}
\begin{lstlisting}[language=PHP, caption=Signed HTTP Request, label=fig:http_signature, float=h]

GET /users/username/inbox HTTP/1.1
Host: mastodon.example
Date: 18 Dec 2019 10:08:46 GMT
Accept: application/activity+json
Signature: keyId="https://my-example.com/actor#main-key",headers="(request-target) host date",signature="Y2FiYW...IxNGRiZDk4ZA=="

\end{lstlisting}

For Mastodon to be able to implement these signatures, it was necessary to generate keypairs for the users. For this, Mastodon added a new property \emph{publicKey} to the actor object, which includes the pem-formatted public key of an RSA-2018 keypair, as shown in figure\ref{fig:mastodon_actor_object}.

\lstset{style=JSONStyle}
\begin{lstlisting}[language=PHP, caption=Mastodon extended version of ActivityPub's actor object, label=fig:mastodon_actor_object, float=ht]
{
    "@context": [
        "https://www.w3.org/ns/activitystreams",
    ],
    "id": "https://mastodon.social/users/bob",
    "type": "Person",
    "following": "https://mastodon.social/users/bob/following",
    "followers": "https://mastodon.social/users/bob/followers",
    "inbox": "https://mastodon.social/users/bob/inbox",
    "outbox": "https://mastodon.social/users/bob/outbox",
    "featured": "https://mastodon.social/users/bob/collections/featured",
    "featuredTags": "https://mastodon.social/users/bob/collections/tags",
    "preferredUsername": "bob",
    "name": "Bob",
    "summary": "",
    "url": "https://mastodon.social/@bob",
    "manuallyApprovesFollowers": false,
    "discoverable": false,
    "published": "2022-01-18T00:00:00Z",
    "devices": "https://mastodon.social/users/bob/collections/devices",
    "publicKey": {
        "id": "https://mastodon.social/users/bob#main-key",
        "owner": "https://mastodon.social/users/bob",
        "publicKeyPem": "-----BEGIN PUBLIC KEY-----\nMIIBIjANBgkqhkiG9w0BAQEFAAOCAQ8AMIIBCgKCAQEAuWWHabkb/1v/OtK3x3EJ\nvxVaHwwN0PxTbOT/UsDNhu8GynBA2A371rg+BEUFGE/yQ6ljaDcqiQtaAMnuvOpT\nLRUOXu5O5Ct+qBWoLb+Box2NkSWJWc4rP+FXEHM+PTPqJJEplZSKIboCjLijw90V\nCD0hL6nBdXSnNp3CeaiDJfjXx+8MoR7xm4AB4Av0nrrM/eGBd60UM3EqMUVAWYtN\n813hrCmbapi++KY9DE7QpMPUoyOiarvFtl8JpV9HWJ1FrwloT17jJ02smPoCy6iv\nQgDXcAWnMm2PZWuSh/uQ7Y6Iq1ZSCkdQJQwaNsyB6O4BlyPujU2f2wg4nzf5QaBn\nwQIDAQAB\n-----END PUBLIC KEY-----\n"
    },
    "tag": [],
    "attachment": [],
    "endpoints": {
        "sharedInbox": "https://mastodon.social/inbox"
    },
    "icon": {
        "type": "Image",
        "mediaType": "image/jpeg",
        "url": "https://files.mastodon.social/accounts/avatars/107/643/267/140/999/130/original/557f01fa567f8220.jpg"
    }
}
\end{lstlisting}

\subsubsection{Resolving accounts}
As explained in \ref{subsec:ap_workflows}, ActivityPub requires a resolving process when wanting to send a message to a user whose account resides on a different server. For this, Mastodon implemented \emph{Well-Known URIs}\footnote{https://www.rfc-editor.org/rfc/rfc8615.html}, which enable the discovery of information about an origin in well-known endpoints\cite{nottingham_2019}. The two most relevant endpoints are the following: 

\begin{itemize}
  \item \textbf{Web Host Metadata: }
    This endpoint allows the discovery of host information, using a lightweight metadata document format. In this context, \emph{host} refers to the entity in charge of a collection of resources defined by URIs with a common URI host \cite{cook_2011}. It employs the XRD 1.0\footnote{https://docs.oasis-open.org/xri/xrd/v1.0/os/xrd-1.0-os.html} document format, which offers a basic and flexible XML-based schema for resource description. Moreover, it provides two mechanisms for providing resource-specific information, \emph{link templates} and \emph{Link-based Resource Descriptor Documents} (LRDD). On the one hand, link templates require a URI to work, thus avoiding the use of fixed URIs. On the other hand, the LRDD relation type is used to relate LRDD documents to resources or host-meta documents \cite{cook_2011}. In the specific case of the Mastodon implementation, requesting the host-meta endpoint will give us back the \emph{lrdd} link to the Webfinger endpoint, where specific resource information can be found. This is illustrated by figures \ref{fig:host_metadata_response} and \ref{fig:host_metadata_request}.

\lstset{style=JSONStyle}
\begin{lstlisting}[language=PHP, caption=Example Host Metadata request to mastodon.social, label=fig:host_metadata_request, float=ht]
    GET /.well-known/host-meta HTTP/1.1
    Host: mastodon.social
    Accept: application/xrd+xml
\end{lstlisting}

\lstset{style=JSONStyle}
\begin{lstlisting}[language=XML, caption=Example Host metadata response from mastodon.social, label=fig:host_metadata_response, float=h!]
    <?xml version="1.0" encoding="UTF-8"?>
    <XRD xmlns="http://docs.oasis-open.org/ns/xri/xrd-1.0">
      <Link rel="lrdd" template="https://mastodon.social/.well-known/webfinger?resource={uri}"/>
    </XRD>
\end{lstlisting}

\item \textbf{Webfinger:}
Mastodon relies on the Webfinger protocol for the resolving process and its federated functioning \cite{rochko_2020}. Webfinger allows for discovering information about persons or other entities on the Internet using HTTP such as a personal profile address, identity service, telephone number or email. Performing a query to a WebFinger endpoint requires a query component with a resource parameter, which is the URI that identifies the identity that is being looked up. Mastodon employs the \emph{acct}\footnote{https://datatracker.ietf.org/doc/html/rfc7565} URI format, which aims to offer a scheme that generically identifies a user's account with a service provider without requiring a specific protocol. Webfinger's response consists of a \emph{JSON Resource Descriptor} (JRD) Document describing the entity \cite{jones_salgueiro_jones_smarr_2013}. Fig. \label{Webfinger response from mastodon.social} shows an example of the returned JRD document provided by the WebFinger endpoint of the \emph{mastodon.social}\footnote{https://mastodon.social} instance when querying the account \emph{acct:bob@mastodon.social}.

\lstset{style=JSONStyle}
\begin{lstlisting}[language=PHP, caption=HTTP request to Webfinger endpoint, label=Webfinger request, float=h]
  GET /.well-known/webfinger?resource=acct:bob@mastodon.social
  Host: mastodon.social
  Accept: application/xrd+xml
\end{lstlisting}

\lstset{style=JSONStyle}
\begin{lstlisting}[language=PHP, caption=Webfinger response, label=Webfinger response from mastodon.social, float=h]
{
   "subject": "acct:bob@mastodon.social",
   "aliases": [
       "https://mastodon.social/@bob",
       "https://mastodon.social/users/bob"
   ],
   "links": [
       {
           "rel": "http://webfinger.net/rel/profile-page",
           "type": "text/html",
           "href": "https://mastodon.social/@bob"
       },
       {
           "rel": "self",
           "type": "application/activity+json",
           "href": "https://mastodon.social/users/bob"
       },
       {
           "rel": "http://ostatus.org/schema/1.0/subscribe",
           "template": "https://mastodon.social/authorize_interaction?uri={uri}"
       }
   ]
}
\end{lstlisting}
\end{itemize}

\section{Decentralized Identifiers} \label{section:dids}

On July 19. 2022 the W3C announced that Decentralized Identifiers (DIDs) v1.0 is officially a Web standard. This new type of globally unique identifier brings a Self-Sovereign approach to digital identities enabling both individuals and organizations to take greater control of their online information and relationships while also providing greater security and privacy \cite{w3c_2022}. Self-Sovereign Identity \emph{SSI} implies a sovereign, enduring, decentralized, and portable digital identity for any human or non-human entity, that enables its owner to access services in the digital world in a secure, private, and trusted manner. DIDs are the key component of the SSI framework, as they allow identifiers to be created independently of any centralized registry, identity provider, or certificate authority with full control given to its owner\cite{Naik_Jenkins_2021}\cite{sporny_longley_sabadello_reed_steele_2021}. 

\subsection{Architecture}

The DID itself is a URI that consists of 3 different parts. The DID URI scheme identifier, the method identifier and the DID method-specific identifier, as shown in figure \ref{fig:did}. The entity being identified by the DID is called the \emph{DID subject}. \emph{Everything} can be identified by a DID, including any person, group, organization, as well as a physical, digital, or logical thing \cite{Conway_Hughes_Ma_Poole_Riedel_2019}\cite{sporny_longley_sabadello_reed_steele_2021}.

\begin{figure}[h]
  \centering
  \includegraphics[width=0.5\textwidth]{related_work/parts-of-a-did.png}
  \caption{DID composition \cite{sporny_longley_sabadello_reed_steele_2021}}
  \label{fig:did}
\end{figure}


The DID Subject that can modify the DID Document is called the \emph{DID controller}. Usually, the DID subject is the DID controller, however, this is not compulsory. As shown in figure \ref{fig:did_architecture}, a DID resolves to a \emph{DID Document}, which is a JSON-based object that contains information associated with a DID. It includes verification methods, such as cryptographic public keys, as well as services that are relevant to be able to interact with the DID Subject. An example of a DID Document can be seen in \ref{fig:did_document}. Furthermore, it is possible to retrieve a specific resource of the DID document by using a \emph{DID URL}, which is a DID that includes a path, query, or fragment \cite{sporny_longley_sabadello_reed_steele_2021}.

% ---- DID Document -----
\lstset{style=JSONStyle}
\begin{lstlisting}[language=PHP, caption=Example DID Document, label=fig:did_document]

{
  "@context": "https://w3id.org/did/v1",
  "id": "did:example:123456789abcdefghi",
  "publicKey": [{
    "id": "did:example:123456789abcdefghi#keys-1",  // DID URL 
    "type": "RsaVerificationKey2018",
    "owner": "did:example:123456789abcdefghi",
    "publicKeyPem": "..."
  }],
  "authentication": [{
    "type": "RsaSignatureAuthentication2018",
    "publicKey": "did:example:123456789abcdefghi#keys-1"
  }],
  "service": [{
    "type": "ExampleService",
    "serviceEndpoint": "https://example.com/endpoint/8377464"
  }]
}

% \end{lstlisting}
 
DID documents are stored in a \emph{Verifiable Data Registry} (VDR). This is essentially any system that enables capturing DIDs and returning required data to generate DID documents, such as distributed ledgers, decentralized file systems, any type of database, peer-to-peer networks, or other types of trustworthy data storage. Our next component is the \emph{DID method}, which describes the processes for CRUD operations for DIDs and DID documents, based on a specific type of VDR. According to the DID registry\footnote{https://www.w3.org/TR/did-spec-registries/\#did-methods} of the W3C, there are around 103 registered DID method specifications. More information about the different existing DID methods can be found in the next \autoref{subsec:did_methods}. \\
The last major component in this architecture overview is the one in charge of resolving DIDs is called \emph{DID resolver}. This component implements the \emph{DID resolution}, which consists of taking a DID as an input, and giving a DID Document as an output \cite{sporny_longley_sabadello_reed_steele_2021}. The Identifiers \& Discovery Working Group (ID WG) has implemented a prototype Universal Resolver\footnote{https://github.com/decentralized-identity/universal-resolver}, which allows the resolution of DIDs for numerous DID methods. In addition, this working group has also developed a Universal Registrar\footnote{https://uniregistrar.io/}, which allows the creation, edition, and deactivation of the DIDs across different DID methods.

\begin{figure}[H]
  \centering
  \includegraphics[width=0.8\textwidth]{related_work/did_brief_architecture_overview.png}
  \caption{DID architecture overview \cite{sporny_longley_sabadello_reed_steele_2021}}
  \label{fig:did_architecture}
\end{figure}

\subsection{DID methods}\label{subsec:did_methods}

 Based on their characteristics and patterns, DIDs can be sorted into different categories \cite{preukschat_reed_2021}, for example: 

\begin{itemize}
  \item \textbf{Ledger-based DIDs}: This includes all the DIDs that store DIDs in a blockchain or other Distributed Ledger Technologies (DLTs). Examples include \emph{did:btcr}, \emph{did:ethr} and \emph{did:trx}, whose DIDs are stored in the Bitcoin, Ethereum, and Tron network correspondingly.
  \item \textbf{Ledger Middleware (\emph{Layer 2}) DIDs}: Layer 2 refers to a framework or protocol that is built on top of an existing blockchain system that takes the transactional burden away from layer 1, making it more scalable \cite{weston_2022}. An example DID method in this category is the \emph{did:ion}\footnote{https://identity.foundation/ion/}, which runs in a layer on top of Bitcoin. 
  \item \textbf{Peer DIDs}: DIDs have the required ability to be resolvable, however not all of them have to be globally resolvable. The DIDs in this category do not exist on a global source of truth but in the context of relationships between peers in a limited number of participants. Nonetheless, they are still valid DIDs as they comply with the core properties and functionalities that a DID has to provide. \cite{preukschat_reed_2021}.
  \item \textbf{Static DIDs}: This type of DIDs are limited in the kind of operations that can be performed on them. These DIDs are not stored in any VDR, consequently, it is not possible to update, deactivate or rotate them. Using the \emph{did:key} method as an example, the DID-method-specific part of the DID is encoded in a way that the DID document can be extracted from it.\cite{longley_zagidulin_sporny_2022}.
\end{itemize}

There are more DIDs that do not necessarily fall into these categories, however, the majority of the existing ones do. Before arriving at the end of the DID section in this thesis, 

% did:ethr
% This DID method was registered by Uport, an Ethereum-based system for self-sovereign identity \cite{Lundkvist_Heck_Torstensson_Mitton_Sena_2016}. It allows any Ethereum smart contract or key pair account, or any secp256k1 public key to becoming a valid identifier \cite{nistor_grassberger_carlin_2022}. UPort is a smart-contract-based system that abstracts user accounts using proxy contracts that are held by the user but may be retrieved if keys are lost. To do this, controller contracts create trusted entities for asserting ownership \cite{8783188}


% did:ens:
% The Ethereum Name Service (ENS) is a distributed, open, and extensible naming system based on the Ethereum blockchain. Its purpose is to map names like 'bob.eth' into machine-readable identifiers like Ethereum addresses, other cryptocurrency addresses, content hashes, and metadata. In the same way the Domain Name System (DNS) maps human-readable domains like 'google.com' to IP addresses. However, their architecture differs due to the capabilities and constraints of the Ethereum blockchain. (https://github.com/veramolabs/did-ens-spec )
% This DID method was registered by Consensys, a blockchain venture studio founded by Ethereum's cofounder Joseph Lubin that supports Ethereum-based projects. Some of these projects include uPort, Metamask, Civil, Truffle and Gnosis. [https://decrypt.co/resources/consensys]. 


% This DID method was created with the following objectives, firstly to integrate ENS names to be interoperable with applications that implement DIDs, and secondly, to add DID-capabilities to ENS names such as services and verification methods [https://github.com/veramolabs/did-ens-spec].

% did:ion:
% This DID method works by registering DIDs to the Bitcoin network. Launched to the public in March 2021, Identity Overlay Network (ION) is Microsoft's DID-network built on top of the Bitcoin Network implemented using Sidetree (https://techcommunity.microsoft.com/t5/identity-standards-blog/ion-we-have-liftoff/ba-p/1441555 ). 
% ION is public, permissionless, it does not implement its own currency and its nodes do not require an additional consensus mechanism. (https://github.com/decentralized-identity/ion ).





% Comparing DIDs with other identifiers


% UUID Universally Unique Identifiers: 
% UUIDs are similar to DIDs, because they do not require a centralized registration authority. But they are not resolvable or cryptographically-verifiable. 


% Almost all types of identifier systems can support DIDs. This would create an interoperability bridge between centralized, federated and decentralized worlds. Implementers can use any kind of computing infrastructure they trust, such as distributed ledgers, decentralized file systems, distributed databases or peer-to-peer networks. 


% Services
% Services are used to specify ways to communicate with the DID subject. Services can be decentralized identity management for further discovery, authentication, authorization or interaction. 


% DID Privacy 

% The 7 Principles of Privacy by design have been applied to DIDs. (https://iapp.org/media/pdf/resource_center/pbd_implement_7found_principles.pdf )






%  A DID is unlikely to contain any information about the DID subject, so further information about the DID subject is only discoverable by resolving the DID to the DID document, obtaining a verifiable credential about the DID, or via some other description of the DID.



\section{DIDComm Messaging}\label{section:didcomm}

The Hyperledger Foundation is an open-source collaborative effort intended to further develop blockchain technologies across industries \cite{jones_boswell_2022}. Started in 2016 by the Linux Foundation, it has given birth to numerous enterprise-grade software open-source projects that can be classified into DLTs, libraries, tools and labs \cite{lusard_lehors_muscara_boswell_zsigri_2021}. One of these graduate projects is Hyperledger Aries, which together with Hyperledger Indy (HI) and Hyperledger Ursa (HU), makes up the Sovereign Identity Blockchain Solutions of Hyperledger. HI supplies a distributed ledger specifically built for decentralized identity, HU is a shared cryptography library that helps to avoid duplicating cryptographic work across projects while also potentially increasing security. Finally, Aries provides solutions for SSI-based identity management, including key management, credential management, and an encrypted, peer-to-peer DID-based messaging system that is now labeled as DIDComm v1 \cite{jones_boswell_2022}. 
Based on DIDComm v1, the Communication Working Group (CWG) of the DIF has implemented DIDComm v2. The CWG pursues the standardization of DIDComm v2 not only to widen its implementation beyond Aries-based projects but to create an interoperable layer that would allow higher-order protocols to build upon its security, privacy, decentralization, and transport independence in the same way web services build upon HTTP. \cite{young_2020} \cite{curren_looker_terbu_2020}

From this point on, the term \emph{DIDComm} will refer to DIDComm Messaging v2. DIDComm can be described as a communication protocol that promises a secure and private methodology that builds on top of the decentralized design of DIDs. It is a versatile protocol that supports a wide range of features, such as security, privacy, decentralization, routable, interoperability, and the ability to be transport-agnostic \cite{curren_looker_terbu_2020}.

To better understand how it works, let's look at how it would work in a scenario where Alice wants to send a private message to Bob: 

\begin{algorithm}[H]
\caption{Example of DID communication using DIDComm \cite{Abramson_2020}}
\label{alg:didcomm_example}
  \begin{algorithmic}[1]
    \State Alice has a private key \emph{sk\textsubscript{a}} and a DID Document for Bob containing an endpoint \emph{(endpoint\textsubscript{bob})} and a public key \emph{(pk\textsubscript{b})}.

    \State Bob has a private key \emph{sk\textsubscript{b}} and a DID Document for Alice containing her public key \emph{(pk\textsubscript{a})}.

    \State Alice encrypts plaintext message \emph{(m)} using pk\textsubscript{b} and creates an encrypted message \emph{(eb)}.

    \State Alice signs eb using her private key \emph{sk\textsubscript{a}} and creates a signature \emph{(s)}.

    \State Alice sends \emph{(eb, s)} to \emph{endpoint\textsubscript{bob}}.

    \State Bob receives the message from Alice at \emph{{endpoint\textsubscript{bob}}.}

    \State Bob verifies \emph{(s)} using Alice's public key \emph{pk\textsubscript{a}}

    \If{Verify \emph{(eb, s, pk\textsubscript{a})} = 1}
    \State Bob decrypts \emph{eb} using \emph{sk\textsubscript{b}}.
    \State Bob reads the plaintext message \emph{(m)} sent by Alice
    \EndIf
  \end{algorithmic}
\end{algorithm}


% [https://arxiv.org/pdf/2006.02456.pdf ]

DIDComm differs from the current dominant web paradigm, where something as simple as an API call requires an almost immediate response through the same channel from the receiving end. This duplex request-response interaction is, however, not always possible as many agents may not have a constant network connection or may interact only in larger time frames, or may even not listen over the same channel where the original message was sent. DIDComm's paradigm is asynchronous and \emph{simplex}. Thus showing a bigger resemblance with the email paradigm. Furthermore, the web paradigm goes under the assumption that traditional methods for processes like authentication, session management, and end-to-end encryption are being used. Didcomm does not require certificates from external parties to establish trust, nor does it require constant connections for end-to-end transport-level encryption (TLS). Taking the security and privacy responsibility away from institutions and placing it with the agents. All of this without limiting the communication possibilities because of its ability to function as a base layer where opposed capabilities like sessions and synchronous interactions can be built upon. \cite{curren_looker_terbu_2020}

To achieve the encryption and signing processes mentioned in algorithm \ref{alg:didcomm_example}, Didcomm implements a family of the Internet Engineering Task Force (IETF) standards, collectively called JSON Object Signing and Encryption (JOSE), which will be further explained in the next section. 

\subsection{JOSE Family}

This RFC family includes both the JSON Web Signature (JWS) and the JSON Web Encryption (JWE) standards that are subclasses of the JSON Web Token (JWT), and JSON Web Key (JWK).


The second part of the DIDComm enablement, namely the signing and encrypting, will depend on what kind of security guarantees we want to achieve. DIDComm offers different approaches based on signing and two types of message encryption. Authenticated Sender Encryption (\emph{authcrypt}) and Anonymous Sender Encryption (\emph{anoncrypt}) are both encrypted and delivered to the recipient's DID. Still, they differ because only \emph{authcrypt} gives direct guarantees about the sender's identity. Sending anonymous messages in social networks is not usually the case, and removing the attribution of a post can lead to other problems \cite{martin_2022}. However, social networks like Ask.fm\footnote{https//ask.fm} or NGL\footnote{https://ask.fun} that rely on anonymous posts could use the advantages of \emph{anoncrypt}.

DIDComm recommends using \emph{authcrypt} as the standard to provide confidentiality, message integrity, and authenticity of the sender. For this, \emph{authcrypt} requires the public key authenticated encryption algorithm \emph{ECDH-1PU}\footnote{https://datatracker.ietf.org/doc/html/draft-madden-jose-ecdh-1pu-04}. This algorithm j

This is due to the use of the 
Elliptic-curve Diffie\-Hellman (ECDH) protocol, which allows two parties to build a secure and private channel across an insecure and observable network. This protocol is part of the foundation of popular messaging apps such as Facebook Messenger, Whatsapp, Signal, and Didcomm V1 \cite{shaaban_2021}. This key agreement protocol works by having both parties interchanging the public keys of their EC keypairs and some other public information. Using this public data and their private keys both parties can calculate a shared secret value, which it's the same for both parties. Any other observer or third party is not able to compute this shared secret without the private data \cite{5972416}. 

An alternative to \emph{Authcrypt} that also complies with the required confidentiality and non-repudiation requirements is to have a nested JWT. To achieve this, the plaintext is first signed and then the resulting JWS is used as the payload of a JWE. The algorithm \ref{alg:nested_jwt} illustrates better the workings of this. 

\begin{algorithm}[H]
  \caption{Communication example with nested JWT}
  \label{alg:nested_jwt}
    \begin{algorithmic}[1]
      \State Alice signs a plain text message using her private key \emph{sk\textsubscript{a}} and creates a \emph{(jws)}.
  
      \State Alice encrypts the \emph{(jws)} using Bob's public key pk\textsubscript{b} and creates a \emph{(jwe)}.
  
      \State Alice sends \emph{jwe} to Bob.
  
      \State Bob decrypts \emph{(jwe)} using his private key \emph{sk\textsubscript{b}} and obtains the \emph{jws}
      \State Bob verifies \emph{(jws)} using Alice's public key \emph{pk\textsubscript{a}}
  \end{algorithmic}
\end{algorithm}
  
In both alternatives, the identity of the sender can be confirmed and the plain text message remains encrypted and hidden from any third party that might manage to read the payload. As mentioned in \autoref{section:didcomm}, the JWE can be sent through any unprotected protocol and still keep all of its advantages. 



Compare with other encryption methods. (TLS, Whatsapp end-to-end, Signal) (Use image below)
 

% ---Left out well-known URIs. ----------
% 
% 
% \item \textbf{Change password:}
%   This endpoint origin from the proposal of the Web Application Security Working Group (WebAppSec) of the W3C. It defines a well-known URL that sites can use to make their change password forms discoverable by tools. This URL would enable software, especially password managers, to easily redirect users to the right link for them to change their password \cite{mondello_o'connor_2021}.
% 
%   \item \textbf{NodeInfo: }
%   NodeInfo is an initiative to standardize the presentation of metadata about a server operating one of the distributed social networks. The two main aims are to get greater insights into the distributed social networking user base and to provide tools that allow users to select the most suited software and server for their requirements. Mastodon is one of the implementers of this protocol, along with other federated social networks such as Diaspora, Peertube and WordPress [http://nodeinfo.diaspora.software/ ]. NodeInfo specifies that servers must provide the well-known path /.well-known/nodeinfo and provide a JRD document referencing the supporting documents via Link elements, as shown in \label{NodeInfo response example}. [http://nodeinfo.diaspora.software/protocol.html ] Accessing the hypertext reference from the JRD response will give a schematized series of metadata of the instance running the endpoint, such as NodeInfo schema version, software, protocols supported by the server, statistics and even a list of third-party services that can interact with the server via an API. Figure \label{fig:nodeinfo_response} shows the NodeInfo 2.0 schema of the Mastodon instance mastodon.social. 
% \end{itemize}
% 
% 
% \lstset{style=JSONStyle}
% \begin{lstlisting}[caption=NodeInfo response example, label=NodeInfo response example, float=h]
%   {
%       "links": [
%           {
%             "rel": "http://nodeinfo.diaspora.software/ns/schema/2.1",
%               "href": "https://example.org/nodeinfo/2.1"
%           }
%       ]
%   }
% \end{lstlisting}

% \lstset{style=JSONStyle}
% \begin{lstlisting}[language=PHP, caption=NodeInfo request]
%     GET /wel-known/nodeinfo/2.0 HTTP/1.1
%     Host: mastodon.social
% \end{lstlisting}

% \lstset{style=JSONStyle}
% \begin{lstlisting}[language=PHP, caption=NodeInfo response for mastodon.social, label=fig:nodeinfo_response, float=h]
% {
%    "version": "2.0",
%    "software": {
%        "name": "mastodon",
%        "version": "3.5.3"
%    },
%    "protocols": [
%        "activitypub"
%    ],
%    "services": {
%        "outbound": [],
%        "inbound": []
%    },
%    "usage": {
%        "users": {
%            "total": 718555,
%            "activeMonth": 61717,
%            "activeHalfyear": 178356
%        },
%        "localPosts": 36484017
%    },
%    "openRegistrations": true,
%    "metadata": []
% }
% \end{lstlisting}